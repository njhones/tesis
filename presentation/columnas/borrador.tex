%\newtheorem{theorem}{Teorema}[section]
%\setbeamercolor{normal text}{fg=black,bg=mylightgrey}  


\documentclass[notes=show]{beamer}
%%%%%%%%%%%%%%%%%%%%%%%%%%%%%%%%%%%%%%%%%%%%%%%%%%%%%%%%%%%%%%%%%%%%%%%%%%%%%%%%%%%%%%%%%%%%%%%%%%%%%%%%%%%%%%%%%%%%%%%%%%%%%%%%%%%%%%%%%%%%%%%%%%%%%%%%%%%%%%%%%%%%%%%%%%%%%%%%%%%%%%%%%%%%%%%%%%%%%%%%%%%%%%%%%%%%%%%%%%%%%%%%%%%%%%%%%%%%%%%%%%%%%%%%%%%%
\usepackage{mathpazo}
\usepackage{hyperref}
\usepackage{multimedia}
\usepackage{graphics}
\usepackage{graphicx}
\usepackage{color}
\usepackage{pgf}

%TCIDATA{OutputFilter=LATEX.DLL}
%TCIDATA{Version=5.50.0.2890}
%TCIDATA{<META NAME="SaveForMode" CONTENT="1">}
%TCIDATA{BibliographyScheme=Manual}
%TCIDATA{Created=Thursday, June 14, 2007 23:33:38}
%TCIDATA{LastRevised=Wednesday, June 20, 2007 05:51:14}
%TCIDATA{<META NAME="GraphicsSave" CONTENT="32">}
%TCIDATA{<META NAME="DocumentShell" CONTENT="Other Documents\SW\Slides - Beamer">}
%TCIDATA{CSTFile=beamer.cst}

\setbeamercovered{highly dynamic}
\newcommand{\myblue}{\only{\color{blue}}}
\newtheorem{claim}{Caracter\'isticas}[section]
\newenvironment{stepenumerate}{\begin{enumerate}[<+->]}{\end{enumerate}}
\newenvironment{stepitemize}{\begin{itemize}[<+->]}{\end{itemize} }
\newenvironment{stepenumeratewithalert}{\begin{enumerate}[<+-| alert@+>]}{\end{enumerate}}
\newenvironment{stepitemizewithalert}{\begin{itemize}[<+-| alert@+>]}{\end{itemize} }
\usetheme{CambridgeUS}
\input{tcilatex}
\begin{document}

\title[Crecimiento y decrecimiento de pol\'igonos]{Crecimiento y
decrecimiento de pol\'igonos mediante paralelas}
\author[Jhones]{Nelson Gonz\'alez Jhones}
\institute[MATCOM UH]{\\
Universidad de la Habana Facultad de Matem\'atica y Computaci\'on}
\date[06/07]{Junio 2007}

%\begin{figure}[h]
%\centering
%\includegraphics[height=2cm, width=1cm ]{C:/c/escudo.jpg}%j	   
%\end{figure}
%\qquad \qquad \qquad {\titlepage}

%\begin{center}
%\vspace{1cm} 
%\end{center}


%\subsection{Conclusiones}


\begin{frame}
\frametitle{Resultados}

\begin{block}{}
\begin{itemize}
	\item<1-| alert@+>La eliminaci\'on local provoca la eliminaci\'on de segmentos que deberian aparecer en el resultado
	\item<2-| alert@+>Eficiente para distancias peque\~nas.
	\item<3-| alert@+>Intersecciones con la original parcial y completamente.  
\end{itemize}
\end{block}

\transsplitverticalout[duration=0.4]
\end{frame}

\begin{frame}
\begin{figure}
	\centering
		\includegraphics<1>[height=4cm, width=5cm]{C:/c/ar2.png}%j		
		\includegraphics<2>[height=4cm, width=5cm]{C:/c/arbol1.png}%j	 	
		\caption{\only<1>{p}\only<2>{s}}
\end{figure}
\end{frame}


\end{document}
